\documentclass[12pt]{article}
\usepackage{amsmath}
\usepackage{booktabs}
\usepackage{pgfplotstable}
\usepackage{graphicx}
\title{DTL \LaTeX\ Assignment}
\date{2022-22-11}
\author{Preet Trivedi 112103160}
\begin{document}
\maketitle
\newpage
\tableofcontents
\newpage
\section{Mathematics paper}
\subsection{Details}
\begin{center}
\large\textbf{MA-19002 Linear Algebra}
\end{center}
\hrule

\begin{flushright}
 Program: S.Y.B.Tech \hfill Academic Year: 2022-23 
\\Examination: End Semester Examination \hfill Maximum marks: 60
\\Date: 30/11/2022 \hfill Time: 10:00 am - 1:00 pm
\\\hfill \textbf{Student MIS Number:}\\ %table insert
\end{flushright}
\begin{table}[h!]
\begin{flushright}
\begin{tabular}{|c|c|c|c|c|c|c|c|c|}
\hline
 & & & & & & & &  \\
\hline
\end{tabular}
\end{flushright}
\end{table}
\vspace{4cm}
\subsection{Instructions}
\begin{enumerate}
\item Write your MIS number on Question Paper.
\item Writing anything on question paper is not allowed.
\item Mobile phones and programmable calculators are strictly prohibited.
\item Exchange/Sharing of stationery, calculator,etc is not allowed.
\item Figures to the right indicate the course outcomes and maximum marks.
\item\textbf{Answers to all subparts should be written together.}
\end{enumerate}
\newpage
\begin{flushleft}
\large\textbf{Question [I]}
\end{flushleft}
Attempt \textbf{any two}:
\begin{enumerate}
\item Find the vector parallel to the line of intersection of the two planes: $$2x-y+z=1;$$ $$3x+y+z=2$$ \hfill [CO1][2]
\item Find eigenvalues and corresponding eigenvectors of  $$ A= \begin{bmatrix}
2 & -1 \\
-1 & 2 
\end{bmatrix} $$
Hence, find an orthogonal basis for $R^{2}$ \hfill [CO3][3]
\end{enumerate}
\vspace{1cm}
\begin{flushleft}
\large\textbf{Question [II]}
\end{flushleft}
Attempt the following:
\\1. Find the rank of matrix 
$$ \begin{bmatrix}
8 & 6 & 4 & 1 & 3 \\
2 & 1 & -7 & 4 & 1 \\
1 & 1 & -1 & 2 & 1 \\
1 & -1 & 2 & 0 & 0 
\end{bmatrix} $$
\begin{center}
\textbf{OR}
\end{center}
2. Test the convergence of improper integral $$ \int_{\pi}^{\infty}     \frac{2+\cos x}{x} \, dx $$ \hfill [CO3][3]
\\3. Assume that the given series is convergent and find its limit. $$\sqrt{1},\sqrt{1+\sqrt{1}},\sqrt{1+\sqrt{1+\sqrt{1}}},...$$ \hfill [CO4][2]
\\\begin{flushright}
\textbf{P.T.O.}
\end{flushright}
\newpage
\begin{flushleft}
\large\textbf{Question [III]}
\end{flushleft}
Discuss the convergence of \textbf{any one} of the series: \hfill [CO5][2]
$$1-\frac{1}{3}+\frac{1}{2}-\frac{1}{6}+\frac{1}{3}-\frac{1}{9}+\frac{1}{4}-\frac{1}{12}+ \cdots$$
\begin{center}
\textbf{OR}
\end{center}
$$\sum_{}^{}(-1)^{n}b_{n}\ if \ b_{n}= \begin{cases}
\frac{1}{n^{2}} & \text{if $n$ is odd}
\\\frac{1}{2^{n}} & \text{if $n$ is even}
\end{cases} $$
\begin{flushleft}
\large\textbf{Question [IV]}
\end{flushleft}
1. Use Lagrange's mean value theorem to prove the inequality $$|tan^{-1}\alpha - tan^{-1}\beta|\le |\alpha-\beta|$$ for all real numbers $\alpha$ and $\beta$. \hfill [CO2][2]
\\\\2. Evaluate \textbf{any two} of the following: (\textbf{write the answer in the simplest form}) \hfill[CO3]$[2\times2.5=5]$
\\\\(a)$\int_{0}^{\infty} \sqrt[4]{x}e^{-\sqrt{x}}\,dx$
(b)$\int_{0}^{\frac{\pi}{2}} \ (\sqrt{\tan x}+ \sqrt{\sec x}) \ dx$
(c)$\int_{0}^{1} \ \sqrt{x^{3}} \sqrt{1-\sqrt{x}}\ dx$
\vspace{2cm}\hrule
\begin{center}
\textbf{ALL THE BEST!}
\end{center}
\newpage
\section{Tables}
\subsection{Using \LaTeX\ }
\begin{table}[h!]
\begin{center}
\begin{tabular}{|c|c|c|c|}
\hline
\textbf{\large Marks} & \multicolumn{2}{c|}{\textbf{\large Number of students}} & \textbf{\large Total} \\
\cline{2-3}
  & \textbf{Males} & \textbf{Females} & \\
  \hline
  30-40 & 8 & 16 & 14 \\
  \hline
  40-50 & 16 & 10 & 26 \\
  \hline
  50-60 & 14 & 16 & 30 \\
  \hline
  60-70 & 12 & 8 & 20 \\
  \hline
  70-80 & 6 & 4 & 10 \\
  \hline
  Total & 56 & 44 & 100 \\
  \hline
\end{tabular}
\caption{This is a table.}
\label{Table 1.1}
\end{center}
\end{table}
\subsection{Using pgfplotstable}
\begin{table}[h!]
  \begin{center}
    \label{table1}
    \pgfplotstabletypeset[
      multicolumn names, 
      col sep=comma,
      every head row/.style={before row=\toprule, after row=\midrule},
        every last row/.style={after row=\bottomrule}
    ]{timetable.csv}
  \end{center}
  \caption{Autogenerated table from .csv file.}
\end{table}
\newpage
\section{Basic Electrical Engineering Paper}
\subsection{Using \textit{graphicx} environment}
\textbf{Q.1.a} Calculate the current through the 6 $\Omega$ resistance shown in \ref{Fig 1}, using Norton's theorem.
\begin{figure}[h]
\centering
\includegraphics[scale=2]{../../Downloads/image0.jpeg} 
\caption{1A}
\label{Fig 1}
\end{figure}
\\\textbf{Q.1.b} Find $V_{1}$, $V_{2}$ and $V_{3}$ for the circuit shown in \ref{Fig 2} by Nodal Analysis.
\begin{figure}[h]
\centering
\includegraphics[scale=2]{../../Downloads/image1.jpeg} 
\caption{1B}
\label{Fig 2}
\end{figure}
\newpage
\twocolumn
\section{IEEE Format Paper}
Artificial Intelligence (AI) has grown dramatically and becomes more and more institutionalized in the 21st Century. In this era of interdisciplinary science, of computer science, cybernetics, automation, mathematical logic, and linguistics \cite{Metev}, questions have been raised about the specific concept of AI \cite{Brekling}. In 1950, Turing \cite{Zhang} presented the famous “Turing Test” which defined of the concept of “Machine Intelligence”. On this background, the origins of AI can be traced to the workshop held on the campus of Dartmouth College in 1965 \cite{Wegmuller}, in which McCarthy persuaded participants to accept the concept of “Artificial Intelligence”. It is likewise the beginning of the first “Golden age” of AI.Given below in \ref{Fig ai1} shows components of AI.
\begin{figure}[h]
\includegraphics[scale=0.75]{ai.png}
\caption{Components of AI} 
\label{Fig ai1}
\end{figure} 
In simple terms, AI aims to extend and augment the capacity and efficiency of mankind in tasks of remaking nature and governing the society through intelligent machines, with the final goal of realizing a society where people and machines coexist harmoniously together \cite{Sorace}. Due to the historical development, AI has been utilized into several major subjects including computer vision, natural language processing, the science of cognition and reasoning, robotics, game theory, and machine learning since the 1980s \cite{website},\cite{Shell}. These subjects developed independently of each other.
\begin{figure}[h]
\includegraphics[scale=0.4]{The-Brain-Intelligent-Machines-AI-timeline.jpg} 
\caption{History of AI} 
\label{Fig ai2}
\end{figure} 
 However, these disciplines basically had already abandoned the logical reasoning and heuristic search-based methods which were proposed 30 years ago as shown in \ref{Fig ai2}.
\newpage
\begin{thebibliography}{}
\bibitem{Metev} S. M. Metev and V. P. Veiko, Laser Assisted Microtechnology, 2nd ed.,
R. M. Osgood, Jr., Ed. Berlin, Germany: Springer-Verlag, 1998.
\bibitem{Brekling} J. Breckling, Ed., The Analysis of Directional Time Series:
Applications to Wind Speed and Direction, ser. Lecture Notes in
Statistics. Berlin, Germany: Springer, 1989, vol. 61
\bibitem{Zhang} S. Zhang, C. Zhu, J. K. O. Sin, and P. K. T. Mok, “A novel ultrathin
elevated channel low-temperature poly-Si TFT,” IEEE Electron Device
Lett., vol. 20, pp. 569–571, Nov. 1999.
\bibitem{Wegmuller}M. Wegmuller, J. P. von der Weid, P. Oberson, and N. Gisin, “High
resolution fiber distributed measurements with coherent OFDR,” in
Proc. ECOC’00, 2000, paper 11.3.4, p. 109.
\bibitem{Sorace} R. E. Sorace, V. S. Reinhardt, and S. A. Vaughn, “High-speed digitalto-RF converter,” U.S. Patent 5 668 842, Sept. 16, 1997.
\bibitem{website}(2002) The IEEE website.
\bibitem{Shell}M. Shell. (2002)IEEEtran homepage on CTAN. 
\end{thebibliography}
\pagenumbering{gobble}
\end{document}

